\documentclass[../main.tex]{subfiles}
\begin{document}
\chapter{正则化}
% Regularization
\section{过拟合问题}
% The Problem of Overfitting
拟合结果可能恰好合适:能很好的预测(应用到新的样本中),预测较差可能有两种情况:
\begin{itemize}
    \item \textbf{欠拟合 Underfitting}:不能适应训练集,自然也不能预测准确。
    \item \textbf{过拟合 Overfitting}:很好的适应了训练集,甚至\(J(θ)=0\),但是过于强调拟合原始数据,丢失了算法的本质,导致无法\textbf{泛化 (generalized) }到新的样本。
\end{itemize}

解决这个问题有两个办法:
\begin{itemize}
    \item 减少特征数量:手动剔除;模型选择法。
    \item 正则化
\end{itemize}
另外,还有一些能判断拟合情况的方法。

\section{代价函数}
对于代价函数,如果加入一个惩罚项如\(λθ_3^2, λ比较大\),若要最小化代价函数,那么\(θ_3\)就要比较小 ,如果\(θ_3\)是一个高阶项,所得的函数也也更平滑,缓解了过拟合的问题。


若有很多特征,但是并不知道那些特征需要惩罚来缩小 (shrink),这时可以给所有的特征加上惩罚项(根据惯例,不对\(θ_0\)进行惩罚),称之为\textbf{正则化项 regularization term},其中\(λ\)称为\textbf{正则化参数 Regularization Parameter}:
\[
    J(θ)  = \frac{1}{m}\sum\limits_{i=1}^{m}Cost(h_θ(x^{(i)}), y^{(i)}) + \frac{λ}{2m}\sum\limits_{j=1}^{n}θ_j^2
\]
注意,如果\(λ\)设置的过大,将导致所有的参数都很小,处于欠拟合,结果近似一条平行于\(x\)轴的直线。\\
\begin{remark}
    理解很困难,试验观察一下很有效。
\end{remark}

\section{正则化线性回归}
% Regularized Linear Regression
\paragraph{代价函数}
\[
    J(θ)  = \frac{1}{m}\sum\limits_{i=1}^{m}\sum\limits_{i-1}^{m}(h_θ(x)^{(i)}-y^{(i)})^2 + \frac{λ}{2m}\sum\limits_{j=1}^{n}θ_j^2
\]

\paragraph{梯度下降的正则化}~\\
\begin{algorithm}[H]
    initialization\;
    \While{not convergence}{
    \(θ_0 ← θ_0 - α\frac{1}{m}\sum\limits_{i=1}^m(h_θ(x^{(i)})-y^{(i)})x_j^{(i)}\)\;
    \(θ_j ← θ_j(1 - α\frac{λ}{m}) - α\frac{1}{m}\sum\limits_{i=1}^m(h_θ(x^{(i)})-y^{(i)})x_j^{(i)}\) (j=1,2,\ldots, n)\;
    }
\end{algorithm}

\paragraph{正规方程的正则化}
\[
    θ=\Bigg(X^TX  + λ\underbrace{\begin{bmatrix}
            0 &   &   &   &   \\
              & 1 &   &   &   \\
              &   & 1 &   &   \\
              &   &   & ⋱ &   \\
              &   &   &   & 1 \\
        \end{bmatrix}}_{(n+1)×(n+1)}\Bigg)^{-1}X^Ty
\]
正则化还有一个有点,加上正则项后可以保证可逆。

\section{正则化的逻辑回归模型}
% Regularized Logistic Regression
\paragraph{代价函数}
\[
    J(θ) = \frac{1}{m}\sum\limits_{i=1}^{m}-y^{(i)}\log(h_θ(x^{(i)})) + (y^{(i)}-1)\log(1- h_θ(x^{(i)}))
    + \frac{λ}{2m}\sum\limits_{j=1}^{n}θ_j^2
\]

\paragraph{梯度下降的正则化}形式上和线性回归一样:\\
\begin{algorithm}[H]
    initialization\;
    \While{not convergence}{
    \(θ_0 ← θ_0 - α\frac{1}{m}\sum\limits_{i=1}^m(h_θ(x^{(i)})-y^{(i)})x_j^{(i)}\)\;
    \(θ_j ← θ_j(1 - α\frac{λ}{m}) - α\frac{1}{m}\sum\limits_{i=1}^m(h_θ(x^{(i)})-y^{(i)})x_j^{(i)}\) (j=1,2,\ldots, n)\;
    }
\end{algorithm}

\subsection{正则化高级算法}
在我们提供给高级算法的cost function和gradient function使用正则化就好了。

\end{document}