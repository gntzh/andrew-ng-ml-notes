\documentclass{report}

\usepackage{booktabs}% 三线表
\usepackage{caption}
\usepackage[lined,boxed]{algorithm2e}
\def\lang{zh}
\usepackage{./style/packages}
\setmathfont{Cambria Math} % Office携带
\setlist{nosep} % 清楚列表间距

\usepackage{./style/format}
\usepackage{./style/theme}

\PassOptionsToPackage{outputdir=out}{minted}
\usepackage{./style/code}
\usepackage{./style/box}
% \usepackage{./style/extra/markdown}

% 数学
\usepackage{./style/math/math}
% 定理类环境
\usepackage{./style/math/mathenv}

%% 公式间距
\AtBeginDocument{ % 设置在这里或者\normalsize有效
  \setlength{\abovedisplayskip}{5pt}
  \setlength{\belowdisplayskip}{5pt}
  \setlength{\abovedisplayshortskip}{3pt}
  \setlength{\belowdisplayshortskip}{3pt}
}

% \usepackage{./style/macro}

\PassOptionsToPackage{
  scheme=chinese, heading=true, zihao=-4
}{ctex}
% \usepackage[SHS]{./style/zh}

\PassOptionsToPackage{CJKmath}{xeCJK} % 中文支持,CJKmath公式中直接的中文
\RequirePackage[fontset=fandol]{ctex}

% \addinputpath{./content/}
% graphics path
\graphicspath{{./img}}

\usepackage[minted]{./style/subfile}


\usetikzlibrary{positioning,shapes.geometric}
\usetikzlibrary{external}
\tikzexternalize[prefix=tikz/,optimize command away=\includepdf]
\tcbset{shield externalize} % 屏蔽layer层的externalize, 避免与tcolorbox的冲突

\tikzset{
    neuron/.style={
        thick,
        circle,
        minimum size=6mm,
        inner sep=0.05em,
        draw=#1,
        fill=#1!8,
    },
    neuron/.default=orange,
}

\title{吴恩达机器学习入门课程笔记}
\author{Shoor}
\date{\today}

\begin{document}

% \begin{figure}[H]
%     \centering
%     \begin{tikzpicture}[
%         ->,
%         on grid,
%         node distance=3cm,
%         neuron/.append style={
%             ellipse,
%             minimum height=9mm,
%             minimum width=20mm,
%             font=\footnotesize,
%         }
%     ]
%     \node[neuron=black] (l1-0) at (0, 0) {\(+1\)};
%     \node[neuron=blue] (l1-1) at (0, -1.2) {\(x^{(i)}_1\)};
%     \node[neuron=blue] (l1-2) at (0, -2.4) {\(x^{(i)}_2\)};
%     \node[neuron=black, right=of l1-0] (l2-0) {\(+1\)};
%     \node[neuron, right=of l1-1] (l2-1) {\(z^{(2)}_1→a^{(2)}_1\)};
%     \node[neuron, right=of l1-2] (l2-2) {\(z^{(2)}_1→a^{(2)}_1\)};

%     \node[neuron=black, right=of l2-0] (l3-0) {\(+1\)};
%     \node[neuron, right=of l2-1] (l3-1) {\(z^{(3)}_1→a^{(3)}_1\)};
%     \node[neuron, right=of l2-2] (l3-2) {\(z^{(3)}_1→a^{(3)}_1\)};

%     \node[neuron, right=of l3-1, pin={[pin edge={black, thin}]right:\(h_Θ(x)\)}] (l4-1) {\(z^{(4)}_1→a^{(4)}_1\)};
%     \foreach \v[count=\i from 0] in {magenta,red,cyan}{
%         \draw (l3-\i.east) -> (l4-1.west);
%         \draw[\v, thick] (l2-\i.east) -> (l3-1.west) node[pos=0.5, font=\footnotesize]{\(Θ^{(2)}_{1\i}\)};
%         \draw (l2-\i.east) -> (l3-2.west);
%         \foreach \j in {1,2}
%             \draw (l1-\i.east) -> (l2-\j.west);
%     }
%     \end{tikzpicture}
% \end{figure}

\maketitle
\tableofcontents
\subfile{./content/引言.tex}
\subfile{./content/单变量线性回归.tex}
\subfile{./content/线性代数回顾.tex}
\subfile{./content/Octave教程/index.tex}
\subfile{./content/多变量线性回归.tex}
\subfile{./content/正则化.tex}
\subfile{./content/逻辑回归.tex}
\part{下一个}
\subfile{./content/神经网络:表述.tex}
\subfile{./content/神经网络:学习.tex}
% \begin{figure}[H]
%     \centering
%     \begin{tikzpicture}[on grid, node distance=2.5cm,]
%         \node[neuron=blue] (l1-0) at (0, 0) {\(1\)};
%         \node[neuron=blue] (l1-1) at (0, -1) {\(x_1\)};
%         \node[neuron, right=of l1-1, yshift=0.5, pin={[pin edge={black, thin}]right:\(h_Θ(x)\)}] (l2-1) {};
%         \foreach \v[count=\i from 0] in {-10,-20}
%             \draw (l1-\i .east) -> (l2-1.west) node[pos=0.4]{\(\v\)};
%     \end{tikzpicture}
% \end{figure}
\end{document}